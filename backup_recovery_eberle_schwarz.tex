\documentclass[11pt,a4paper,parskip=half]{scrartcl}
\usepackage{ngerman}
\usepackage[utf8]{inputenc}
\usepackage[colorlinks=false,pdfborder={0 0 0}]{hyperref}
\usepackage{graphicx}
\usepackage{caption}
\usepackage{longtable}
\usepackage{float}
\usepackage{fancyhdr}
\usepackage{textcomp}
\usepackage{geometry}
\usepackage{amsmath}
\usepackage{amssymb}
\usepackage{nameref}
\geometry{a4paper, left=30mm, right=25mm, top=30mm, bottom=35mm} 
\usepackage{listings}
\lstset{breaklines=true, breakatwhitespace=true, basicstyle=\tiny, numbers=left, frame=single}
\setcounter{tocdepth}{2} %no subsubsections in TOC

\title{dbarc: Ausarbeitung Backup \& Recovery}
\author{Yanick Eberle, Pascal Schwarz}
\begin{document}
\maketitle
\vfill
\tableofcontents

\pagestyle{fancy}
\section{Vorbereitung}
\subsection{Kontrolle Dateien}
Die genannten Files liegen bei uns im korrekten Pfad:
\begin{lstlisting}
oracle_stud@hades:/> ls /app/oracle_stud/oradata/${ORACLE_SID}/data/
control01.ctl  control02.ctl  control03.ctl  redo01.log  redo02.log  redo03.log  sysaux01.dbf  system01.dbf  temp01.dbf  undotbs01.dbf  users01.dbf
\end{lstlisting}

\subsection{Noarchivelog Modus}
Wir wechseln gemäss den Angaben in der Aufgabenstellung den Modus unserer Oracle-Instanz:
\begin{lstlisting}
oracle_stud@hades:/> sqlplus sys as sysdba

SQL*Plus: Release 11.2.0.1.0 Production on Fri Jun 21 07:02:06 2013
Copyright (c) 1982, 2009, Oracle.  All rights reserved.

Enter password: 
Connected to an idle instance.

SQL> STARTUP MOUNT
ORACLE instance started.

Total System Global Area  417546240 bytes
Fixed Size		    2213936 bytes
Variable Size		  339740624 bytes
Database Buffers	   67108864 bytes
Redo Buffers		    8482816 bytes
Database mounted.

SQL> ALTER DATABASE NOARCHIVELOG;
Database altered.

SQL> ALTER DATABASE OPEN;
Database altered.
\end{lstlisting}

\subsection{System entlasten}
Bei unserer Instanz liegen die Trace-Files im angegebenen Verzeichnis in einem Unterordner \emph{trace}. Wie die Ausgabe von \emph{find ... | wc -l} zeigt, ist eine grosse Zahl dieser Dateien vorhanden:
\begin{lstlisting}
oracle_stud@hades:/app/oracle_stud/diag/rdbms/dbarc02/dbarc02/trace> find -iname '*.trc' | wc -l
3787
oracle_stud@hades:/app/oracle_stud/diag/rdbms/dbarc02/dbarc02/trace> rm *.trc

\end{lstlisting}

Unser Alert-Log ist mit ca. 100k noch recht kompakt, wir leeren es trotzdem.
\begin{lstlisting}
oracle_stud@hades:...dbarc02/trace> ls -lh alert_dbarc02.log 
-rw-r----- 1 oracle_stud oinstall 103K 2013-06-21 07:08 alert_dbarc02.log
oracle_stud@hades:...dbarc02/dbarc02/trace> echo -n "" > alert_dbarc02.log 
oracle_stud@hades:...dbarc02/trace> ls -lh alert_dbarc02.log 
-rw-r----- 1 oracle_stud oinstall 0 2013-06-21 07:17 alert_dbarc02.log
\end{lstlisting}

Wie der folgende Befehl zeigt, existiert auf dem Server Hades nirgends ein Verzeichnis \emph{bdump}:
\begin{lstlisting}
oracle_stud@hades:/app/oracle_stud/admin/dbarc02> find / -type d -iname 'bdump' 2>/dev/null 
oracle_stud@hades:/app/oracle_stud/admin/dbarc02>
\end{lstlisting}


\section{Export / Import einer Tabelle}
\subsection{Export}
Nach der Anmeldung führen wir gemäss Aufgabenstellung den Export-Befehl im entsprechenden Verzeichnis aus und überprüfen mittels ls, dass die Files erzeugt wurden (da der Datenbank-Name mit unserer SID übereinstimmt, verwenden wir gleich diese exportierte Variable):
\begin{lstlisting}
oracle_stud@hades:/app/oracle_stud/oradata/dbarc02/export> exp scott/tiger FILE=${ORACLE_SID} LOG=${ORACLE_SID}

Export: Release 11.2.0.1.0 - Production on Fri Jun 21 07:23:10 2013

Copyright (c) 1982, 2009, Oracle and/or its affiliates.  All rights reserved.


Connected to: Oracle Database 11g Enterprise Edition Release 11.2.0.1.0 - 64bit Production
With the Partitioning, OLAP, Data Mining and Real Application Testing options
Export done in US7ASCII character set and AL16UTF16 NCHAR character set
server uses AL32UTF8 character set (possible charset conversion)
. exporting pre-schema procedural objects and actions
. exporting foreign function library names for user SCOTT 
. exporting PUBLIC type synonyms
. exporting private type synonyms
. exporting object type definitions for user SCOTT 
About to export SCOTT's objects ...
. exporting database links
. exporting sequence numbers
. exporting cluster definitions
. about to export SCOTT's tables via Conventional Path ...
. . exporting table                           DEPT          4 rows exported
EXP-00091: Exporting questionable statistics.
EXP-00091: Exporting questionable statistics.
. . exporting table                            EMP         14 rows exported
EXP-00091: Exporting questionable statistics.
EXP-00091: Exporting questionable statistics.
. . exporting table                       SALGRADE          5 rows exported
EXP-00091: Exporting questionable statistics.
. exporting synonyms
. exporting views
. exporting stored procedures
. exporting operators
. exporting referential integrity constraints
. exporting triggers
. exporting indextypes
. exporting bitmap, functional and extensible indexes
. exporting posttables actions
. exporting materialized views
. exporting snapshot logs
. exporting job queues
. exporting refresh groups and children
. exporting dimensions
. exporting post-schema procedural objects and actions
. exporting statistics
Export terminated successfully with warnings.


oracle_stud@hades:/app/oracle_stud/oradata/dbarc02/export> ls -lh
total 28K
-rw-r--r-- 1 oracle_stud oinstall  24K 2013-06-21 07:23 dbarc02.dmp
-rw-r--r-- 1 oracle_stud oinstall 1.7K 2013-06-21 07:23 dbarc02.log
\end{lstlisting}

\subsection{Tabelle löschen}
Das folgende Listing zeigt das Löschen der \emph{emp}-Tabelle.

\begin{lstlisting}
oracle_stud@hades:/app/oracle_stud/oradata/dbarc02/export> sqlplus scott/tiger

SQL*Plus: Release 11.2.0.1.0 Production on Fri Jun 21 07:30:04 2013
Copyright (c) 1982, 2009, Oracle.  All rights reserved.

Error accessing PRODUCT_USER_PROFILE
Warning:  Product user profile information not loaded!
You may need to run PUPBLD.SQL as SYSTEM

Connected to:
Oracle Database 11g Enterprise Edition Release 11.2.0.1.0 - 64bit Production
With the Partitioning, OLAP, Data Mining and Real Application Testing options

SQL> DESC emp;
 Name					   Null?    Type
 ----------------------------------------- -------- ----------------------------
 EMPNO					   NOT NULL NUMBER(4)
 ENAME						    VARCHAR2(10)
 JOB						    VARCHAR2(9)
 MGR						    NUMBER(4)
 HIREDATE					    DATE
 SAL						    NUMBER(7,2)
 COMM						    NUMBER(7,2)
 DEPTNO 					    NUMBER(2)

SQL> DROP TABLE emp;
Table dropped.

SQL> DESC emp;
ERROR:
ORA-04043: object emp does not exist
\end{lstlisting}

\subsection{Tabelle zurückladen}
Wir führen den Import-Befehl wiederum gemäss Aufgabenstellung aus. Bei der ersten Frage (\glqq{}Import Data Only?\grqq{}), deren Antwort nicht in der Aufgabe vorgegeben ist, haben wir mit \glqq{}Yes\grqq{} geantwortet. 

Die Beantwortung mit \glqq{}No\grqq{} führt dazu, dass der Import-Befehl die Tabelle nicht erstellt und legiglich versucht, Datensätze einzufügen. Da die Tabelle zuvor aber entfernt wurde, schlägt dies fehl.

\begin{lstlisting}
oracle_stud@hades:/app/oracle_stud/oradata/dbarc02/export> imp scott/tiger

Import: Release 11.2.0.1.0 - Production on Fri Jun 21 07:43:07 2013
Copyright (c) 1982, 2009, Oracle and/or its affiliates.  All rights reserved.
Connected to: Oracle Database 11g Enterprise Edition Release 11.2.0.1.0 - 64bit Production
With the Partitioning, OLAP, Data Mining and Real Application Testing options

Import data only (yes/no): no > 
Import file: expdat.dmp > dbarc02.dmp
Enter insert buffer size (minimum is 8192) 30720> 

Export file created by EXPORT:V11.02.00 via conventional path
import done in US7ASCII character set and AL16UTF16 NCHAR character set
import server uses AL32UTF8 character set (possible charset conversion)

List contents of import file only (yes/no): no > 
Ignore create error due to object existence (yes/no): no > 
Import grants (yes/no): yes > 
Import table data (yes/no): yes > 
Import entire export file (yes/no): no > 
Username: scott
Enter table(T) or partition(T:P) names. Null list means all tables for user
Enter table(T) or partition(T:P) name or . if done: emp
Enter table(T) or partition(T:P) name or . if done: .

. importing SCOTT's objects into SCOTT
. importing SCOTT's objects into SCOTT
. . importing table                          "EMP"         14 rows imported
About to enable constraints...
Import terminated successfully without warnings.
\end{lstlisting}

\subsection{Tabelle prüfen}
Die Tabelle konnte erfolgreich wiederhergestellt werden:

\begin{lstlisting}
oracle_stud@hades:/app/oracle_stud/oradata/dbarc02/export> sqlplus scott/tiger

SQL*Plus: Release 11.2.0.1.0 Production on Fri Jun 21 07:45:46 2013
Copyright (c) 1982, 2009, Oracle.  All rights reserved.
Error accessing PRODUCT_USER_PROFILE
Warning:  Product user profile information not loaded!
You may need to run PUPBLD.SQL as SYSTEM

Connected to:
Oracle Database 11g Enterprise Edition Release 11.2.0.1.0 - 64bit Production
With the Partitioning, OLAP, Data Mining and Real Application Testing options

SQL> DESC emp;
 Name					   Null?    Type
 ----------------------------------------- -------- ----------------------------
 EMPNO					   NOT NULL NUMBER(4)
 ENAME						    VARCHAR2(10)
 JOB						    VARCHAR2(9)
 MGR						    NUMBER(4)
 HIREDATE					    DATE
 SAL						    NUMBER(7,2)
 COMM						    NUMBER(7,2)
 DEPTNO 					    NUMBER(2)
\end{lstlisting}


\section{Export / Import einer Datenbank}
\subsection{Export der Datenbank}
Den Export der gesamtem Datenkbank haben wir mit dem exp-Befehl durchgeführt. Der Output umfasste ingesamt ca. 1300 Zeilen, daher haben wir hier einiges rausgeschnitten. Wir sehen, dass unter Anderem auch die Tabellen \glqq{}dep\grqq{} und \glqq{}emp\grqq{} des Benuzters scott exportiert wurden.
\lstinputlisting{backup_recovery/exp_db_output_cut.txt}

Wir prüfen auch hier, ob die Export-Files gemäss unseren Vorstellungen erstellt wurden. Der Full-Dump ist um ein Vielfaches grösser als der zuvor erstellte Tabellenexport.
\begin{lstlisting}
oracle_stud@hades:/app/oracle_stud/oradata/dbarc02/export> ls -lh
total 42M
-rw-r--r-- 1 oracle_stud oinstall  24K 2013-06-21 07:23 dbarc02.dmp
-rw-r--r-- 1 oracle_stud oinstall  41M 2013-06-21 08:25 dbarc02_full.dmp
-rw-r--r-- 1 oracle_stud oinstall  70K 2013-06-21 08:25 dbarc02_full.log
-rw-r--r-- 1 oracle_stud oinstall 1.7K 2013-06-21 07:23 dbarc02.log
oracle_stud@hades:/app/oracle_stud/oradata/dbarc02/export>
\end{lstlisting}

\subsection{User Scott samt Objekten löschen}
Nach dem Entfernen des Benutzers überprüfen wir mit Hilfe von \emph{all\_users}, ob der Benutzer auch tatsächlich entfernt wurde.
\begin{lstlisting}
oracle_stud@hades:/app/oracle_stud/oradata/dbarc02/export> sqlplus system/manager

SQL*Plus: Release 11.2.0.1.0 Production on Fri Jun 21 08:45:10 2013
Copyright (c) 1982, 2009, Oracle.  All rights reserved.

Connected to:
Oracle Database 11g Enterprise Edition Release 11.2.0.1.0 - 64bit Production
With the Partitioning, OLAP, Data Mining and Real Application Testing options

SQL> DROP USER scott CASCADE;
User dropped.

SQL> select * from all_users;

USERNAME			  USER_ID CREATED
------------------------------ ---------- ---------
MGMT_VIEW			       35 08-MAR-13
SYSMAN				       33 08-MAR-13
APPQOSSYS			       31 08-MAR-13
DBSNMP				       30 08-MAR-13
ORACLE_OCM			       21 08-MAR-13
DIP				       14 08-MAR-13
OUTLN					9 08-MAR-13
SYSTEM					5 08-MAR-13
SYS					0 08-MAR-13

9 rows selected.
\end{lstlisting}

\subsection{User Scott wieder erstellen}
Mit den folgenden Statements haben wir den User Scott neu erstellt und geprüft, ob die Erstellung erfolgreich war:
\begin{lstlisting}
SQL> CREATE USER scott IDENTIFIED BY tiger
  2  DEFAULT TABLESPACE users
  3  QUOTA UNLIMITED ON users;
User created.

SQL> GRANT CONNECT TO scott;
Grant succeeded.

SQL> ALTER USER scott DEFAULT ROLE all;
User altered.

SQL> select * from all_users;
USERNAME			  USER_ID CREATED
------------------------------ ---------- ---------
MGMT_VIEW			       35 08-MAR-13
SYSMAN				       33 08-MAR-13
SCOTT				       36 21-JUN-13
APPQOSSYS			       31 08-MAR-13
DBSNMP				       30 08-MAR-13
ORACLE_OCM			       21 08-MAR-13
DIP				       14 08-MAR-13
OUTLN					9 08-MAR-13
SYSTEM					5 08-MAR-13
SYS					0 08-MAR-13
10 rows selected.
\end{lstlisting}

\subsection{Daten importieren}
Die in folgendem Listing enthaltenen Fehlermeldungen resp. Warnungen (bzgl. dem System-Tablespace) zeigen, dass die Erstellung des Benutzers zuvor noch nicht korrekt war. Mit diesen Warnungen werden zwar die Tabellen erstellt, allerdings fehlen danach die Datensätze.

\begin{lstlisting}
oracle_stud@hades:/app/oracle_stud/oradata/dbarc02/export> imp system/manager file=${ORACLE_SID}_full fromuser=scott touser=scott

Import: Release 11.2.0.1.0 - Production on Fri Jun 21 09:17:32 2013

Copyright (c) 1982, 2009, Oracle and/or its affiliates.  All rights reserved.


Connected to: Oracle Database 11g Enterprise Edition Release 11.2.0.1.0 - 64bit Production
With the Partitioning, OLAP, Data Mining and Real Application Testing options

Export file created by EXPORT:V11.02.00 via conventional path
import done in US7ASCII character set and AL16UTF16 NCHAR character set
import server uses AL32UTF8 character set (possible charset conversion)
. importing SCOTT's objects into SCOTT
. . importing table                         "DEPT"
IMP-00058: ORACLE error 1950 encountered
ORA-01950: no privileges on tablespace 'SYSTEM'
. . importing table                          "EMP"
IMP-00058: ORACLE error 1950 encountered
ORA-01950: no privileges on tablespace 'SYSTEM'
. . importing table                     "SALGRADE"
IMP-00058: ORACLE error 1950 encountered
ORA-01950: no privileges on tablespace 'SYSTEM'
About to enable constraints...
Import terminated successfully with warnings
\end{lstlisting}

Eine Analyse des Problems und des Dump-Files (einfaches anschauen mittels \emph{less}) ergab, dass die Tabellen \emph{dept} und \emph{emp} vor dem Export nicht im Users-Tablespace, sondern im System-Tablespace lagen. Da es ohne Umwege mit dem Utility \emph{imp} nicht möglich ist, die Tabellen des Benutzers in einen anderen Tablespace zu importieren, haben wir dem User Scott ebenfalls Platz auf dem System-Tablespace gegeben:
\begin{lstlisting}
oracle_stud@hades:/app/oracle_stud/oradata/dbarc02/export> sqlplus system/manager

SQL*Plus: Release 11.2.0.1.0 Production on Fri Jun 21 10:04:14 2013
Copyright (c) 1982, 2009, Oracle.  All rights reserved.

Connected to:
Oracle Database 11g Enterprise Edition Release 11.2.0.1.0 - 64bit Production
With the Partitioning, OLAP, Data Mining and Real Application Testing options

SQL> ALTER USER scott QUOTA UNLIMITED ON system;

User altered.
\end{lstlisting}

Diese Anpassung haben wir nach einem erneuten Entfernen und Neu-Erstellung des Benutzers scott vorgenommen. Der Import läuft nun ordnungsgemäss durch:
\begin{lstlisting}
oracle_stud@hades:/app/oracle_stud/oradata/dbarc02/export> imp system/manager file=${ORACLE_SID}_full fromuser=scott touser=scott

Import: Release 11.2.0.1.0 - Production on Fri Jun 21 10:13:12 2013

Copyright (c) 1982, 2009, Oracle and/or its affiliates.  All rights reserved.


Connected to: Oracle Database 11g Enterprise Edition Release 11.2.0.1.0 - 64bit Production
With the Partitioning, OLAP, Data Mining and Real Application Testing options

Export file created by EXPORT:V11.02.00 via conventional path
import done in US7ASCII character set and AL16UTF16 NCHAR character set
import server uses AL32UTF8 character set (possible charset conversion)
. importing SCOTT's objects into SCOTT
. . importing table                         "DEPT"          4 rows imported
. . importing table                          "EMP"         14 rows imported
. . importing table                     "SALGRADE"          5 rows imported
About to enable constraints...
Import terminated successfully without warnings.
\end{lstlisting}

\subsection{User Scott testen}
Nach dem Zurückladen der Daten melden wir uns an und prüfen das Vorhandensein der Tabellen von Scott:
\begin{lstlisting}
oracle_stud@hades:/app/oracle_stud/oradata/dbarc02/export> sqlplus scott/tiger

SQL*Plus: Release 11.2.0.1.0 Production on Fri Jun 21 10:15:08 2013

Copyright (c) 1982, 2009, Oracle.  All rights reserved.

Error accessing PRODUCT_USER_PROFILE
Warning:  Product user profile information not loaded!
You may need to run PUPBLD.SQL as SYSTEM

Connected to:
Oracle Database 11g Enterprise Edition Release 11.2.0.1.0 - 64bit Production
With the Partitioning, OLAP, Data Mining and Real Application Testing options

SQL> desc emp;
 Name					   Null?    Type
 ----------------------------------------- -------- ----------------------------
 EMPNO					   NOT NULL NUMBER(4)
 ENAME						    VARCHAR2(10)
 JOB						    VARCHAR2(9)
 MGR						    NUMBER(4)
 HIREDATE					    DATE
 SAL						    NUMBER(7,2)
 COMM						    NUMBER(7,2)
 DEPTNO 					    NUMBER(2)

SQL> select * from dept;

    DEPTNO DNAME	  LOC
---------- -------------- -------------
	10 ACCOUNTING	  NEW YORK
	20 RESEARCH	  DALLAS
	30 SALES	  CHICAGO
	40 OPERATIONS	  BOSTON

\end{lstlisting}


\section{Cold Backup}
\subsection{DB Stop}
Wir fahren die DB gemäss Anleitung in der Aufgabenstellung herunter:
\begin{lstlisting}
oracle_stud@hades:/app/oracle_stud/oradata/dbarc02/export> sqlplus /nolog

SQL*Plus: Release 11.2.0.1.0 Production on Fri Jun 21 10:20:36 2013

Copyright (c) 1982, 2009, Oracle.  All rights reserved.

SQL> CONNECT / as sysdba
Connected.
SQL> SHUTDOWN IMMEDIATE
Database closed.
Database dismounted.
ORACLE instance shut down.
\end{lstlisting}

Die Kontrolle mittels des ps-Befehls zeigt, dass unsere Instanz (dbarc02) erfolgreich beendet wurde:
\begin{lstlisting}
oracle_stud@hades:/app/oracle_stud/oradata/dbarc02/export> ps -ef | grep smon
1001       641     1  0 Jun14 ?        00:00:14 ora_smon_dbarc01
1001      2907     1  0 May30 ?        00:01:34 ora_smon_dbarc04
1001      2957     1  0 Jun18 ?        00:00:06 ora_smon_dbarc07
1001     12380     1  0 May01 ?        00:01:43 ora_smon_dbarc08
oracle   12826     1  0 Mar01 ?        00:05:11 ora_smon_hades11g
1001     14183     1  0 Mar15 ?        00:02:18 ora_smon_dbarc06
1001     20249     1  0 Mar08 ?        00:06:40 ora_smon_mydb
1001     26856 20395  0 10:22 pts/0    00:00:00 grep smon
oracle   29737     1  0 Mar01 ?        00:04:15 ora_smon_ananke
1001     30687     1  0 Jun07 ?        00:00:39 ora_smon_dbarc05
\end{lstlisting}

\subsection{Sicherung}
Die Sicherung besteht nun aus einem einfachen Datei-Kopiervorgang. Da die Datenbank-Instanz zuvor gestoppt wurde, sind die Dateien in einem konsistenten Zustand.

\begin{lstlisting}
oracle_stud@hades:/app/oracle_stud/oradata/dbarc02/data> cp -v * ../cold_backup/
`control01.ctl' -> `../cold_backup/control01.ctl'
`control02.ctl' -> `../cold_backup/control02.ctl'
`control03.ctl' -> `../cold_backup/control03.ctl'
`redo01.log' -> `../cold_backup/redo01.log'
`redo02.log' -> `../cold_backup/redo02.log'
`redo03.log' -> `../cold_backup/redo03.log'
`sysaux01.dbf' -> `../cold_backup/sysaux01.dbf'
`system01.dbf' -> `../cold_backup/system01.dbf'
`temp01.dbf' -> `../cold_backup/temp01.dbf'
`undotbs01.dbf' -> `../cold_backup/undotbs01.dbf'
`users01.dbf' -> `../cold_backup/users01.dbf'
\end{lstlisting}

\subsection{Neustart, Test, Modifikation}
Da auf der Tabelle ein Primary-Key definiert ist, funktioniert das angegebene Insert-Statement nicht. Wir haben stattdessen mit dem folgenden Statement eine neue Zeile hinzugefügt und die Anzahl Zeilen geprüft.
\begin{lstlisting}
SQL> INSERT INTO "SCOTT"."EMP" (EMPNO, ENAME, JOB, HIREDATE, SAL, DEPTNO) VALUES ('1337', 'TEST', 'BOSS', TO_DATE('23-JAN-82', 'DD-MON-RR'), '1337', '10')

SQL> select count(*) from emp;                   

  COUNT(*)
----------
	15
\end{lstlisting}

\subsection{Fehler provozieren}
Die Tabelle \emph{emp} liegt im \emph{System}-Tablespace, daher benennen wir dieses File um.

Danach tritt beim angegebenen Statement wie erwartet ein Fehler auf (INSERT INTO und SELECT * FROM funktionierten hingegen noch, da die Informationen wohl noch im Cache vorhanden waren).
\begin{lstlisting}
oracle_stud@hades:/app/oracle_stud/oradata/dbarc02/data> sqlplus scott/tiger

SQL*Plus: Release 11.2.0.1.0 Production on Fri Jun 21 10:49:21 2013
Copyright (c) 1982, 2009, Oracle.  All rights reserved.
Error accessing PRODUCT_USER_PROFILE
Warning:  Product user profile information not loaded!
You may need to run PUPBLD.SQL as SYSTEM

Connected to:
Oracle Database 11g Enterprise Edition Release 11.2.0.1.0 - 64bit Production
With the Partitioning, OLAP, Data Mining and Real Application Testing options

SQL> CREATE TABLE aaa AS SELECT * FROM emp;
CREATE TABLE aaa AS SELECT * FROM emp
                                  *
ERROR at line 1:
ORA-00604: error occurred at recursive SQL level 2
ORA-01116: error in opening database file 1
ORA-01110: data file 1: '/app/oracle_stud/oradata/dbarc02/data/system01.dbf'
ORA-27041: unable to open file
Linux-x86_64 Error: 2: No such file or directory
Additional information: 3
\end{lstlisting}

\subsection{Reparatur}
Wir konnten uns aufgrund der folgenden Fehlermeldung nicht mehr normal auf die Instanz verbinden:
\begin{lstlisting}
SQL> CONNECT / AS sysdba
ERROR:
ORA-01075: you are currently logged on
\end{lstlisting}

Informationen von \url{http://bit.ly/btEuNm} haben uns dann allerdings weitergeholfen, so dass wir die Instanz doch noch abwürgen konnten:
\begin{lstlisting}
oracle_stud@hades:/app/oracle_stud/oradata/dbarc02/data> sqlplus -prelim "/as sysdba"

SQL*Plus: Release 11.2.0.1.0 Production on Fri Jun 21 11:00:44 2013

Copyright (c) 1982, 2009, Oracle.  All rights reserved.

SQL> SHUTDOWN ABORT;
ORACLE instance shut down.
\end{lstlisting}

Nach dem Stop der Instanz haben wir alle Files unter Beibehaltung der Berechtigungen (Option \emph{-p} von \emph{cp}) aus dem erstellten Cold-Backup zurückkopiert und die Datenbank neu gestartet:
\begin{lstlisting}
oracle_stud@hades:/app/oracle_stud/oradata/dbarc02/data> rm -v * && cp -pv ../cold_backup/* .
removed `control01.ctl'
removed `control02.ctl'
removed `control03.ctl'
removed `redo01.log'
removed `redo02.log'
removed `redo03.log'
removed `sysaux01.dbf'
removed `system01.dbf_'
removed `temp01.dbf'
removed `undotbs01.dbf'
removed `users01.dbf'
`../cold_backup/control01.ctl' -> `./control01.ctl'
`../cold_backup/control02.ctl' -> `./control02.ctl'
`../cold_backup/control03.ctl' -> `./control03.ctl'
`../cold_backup/redo01.log' -> `./redo01.log'
`../cold_backup/redo02.log' -> `./redo02.log'
`../cold_backup/redo03.log' -> `./redo03.log'
`../cold_backup/sysaux01.dbf' -> `./sysaux01.dbf'
`../cold_backup/system01.dbf' -> `./system01.dbf'
`../cold_backup/temp01.dbf' -> `./temp01.dbf'
`../cold_backup/undotbs01.dbf' -> `./undotbs01.dbf'
`../cold_backup/users01.dbf' -> `./users01.dbf'

oracle_stud@hades:/app/oracle_stud/oradata/dbarc02/data> sqlplus /nolog

SQL*Plus: Release 11.2.0.1.0 Production on Fri Jun 21 11:14:46 2013

Copyright (c) 1982, 2009, Oracle.  All rights reserved.

SQL> CONNECT / as sysdba
Connected to an idle instance.
SQL> STARTUP
ORACLE instance started.

Total System Global Area  417546240 bytes
Fixed Size		    2213936 bytes
Variable Size		  339740624 bytes
Database Buffers	   67108864 bytes
Redo Buffers		    8482816 bytes
Database mounted.
Database opened.
\end{lstlisting}

\subsection{Test Benutzer scott}
Wir können uns nun wieder als Scott verbinden und stellen fest, dass die Tabelle im Zustand zum Zeitpunkt des Cold-Backups wieder verfügbar ist.
\begin{lstlisting}
oracle_stud@hades:/app/oracle_stud/oradata/dbarc02/data> sqlplus scott/tiger

SQL*Plus: Release 11.2.0.1.0 Production on Fri Jun 21 11:16:47 2013
Copyright (c) 1982, 2009, Oracle.  All rights reserved.

Error accessing PRODUCT_USER_PROFILE
Warning:  Product user profile information not loaded!
You may need to run PUPBLD.SQL as SYSTEM

Connected to:
Oracle Database 11g Enterprise Edition Release 11.2.0.1.0 - 64bit Production
With the Partitioning, OLAP, Data Mining and Real Application Testing options

SQL> select count(*) from emp;

  COUNT(*)
----------
	14
\end{lstlisting}


\section{Archive Mode}
yay, eberle


\section{Hot Backup}
yay, eberle


\section{Reflexion}
oh noez, schwarz?


\end{document}