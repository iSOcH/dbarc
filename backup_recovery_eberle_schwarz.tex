\documentclass[11pt,a4paper,parskip=half]{scrartcl}
\usepackage{ngerman}
\usepackage[utf8]{inputenc}
\usepackage[colorlinks=false,pdfborder={0 0 0}]{hyperref}
\usepackage{graphicx}
\usepackage{caption}
\usepackage{longtable}
\usepackage{float}
\usepackage{fancyhdr}
\usepackage{textcomp}
\usepackage{geometry}
\usepackage{amsmath}
\usepackage{amssymb}
\usepackage{nameref}
\geometry{a4paper, left=30mm, right=25mm, top=30mm, bottom=35mm} 
\usepackage{listings}
\lstset{breaklines=true, breakatwhitespace=true, basicstyle=\tiny, numbers=left, frame=single}
\setcounter{tocdepth}{2} %no subsubsections in TOC

\title{dbarc: Ausarbeitung Backup \& Recovery}
\author{Yanick Eberle, Pascal Schwarz}
\begin{document}
\maketitle
\vfill
\tableofcontents

\pagestyle{fancy}
\section{Vorbereitung}
\subsection{Kontrolle Dateien}
Die genannten Files liegen bei uns im korrekten Pfad:
\begin{lstlisting}
oracle_stud@hades:/> ls /app/oracle_stud/oradata/${ORACLE_SID}/data/
control01.ctl  control02.ctl  control03.ctl  redo01.log  redo02.log  redo03.log  sysaux01.dbf  system01.dbf  temp01.dbf  undotbs01.dbf  users01.dbf
\end{lstlisting}

\subsection{Noarchivelog Modus}
Wir wechseln gemäss den Angaben in der Aufgabenstellung den Modus unserer Oracle-Instanz:
\begin{lstlisting}
oracle_stud@hades:/> sqlplus sys as sysdba

SQL*Plus: Release 11.2.0.1.0 Production on Fri Jun 21 07:02:06 2013
Copyright (c) 1982, 2009, Oracle.  All rights reserved.

Enter password: 
Connected to an idle instance.

SQL> STARTUP MOUNT
ORACLE instance started.

Total System Global Area  417546240 bytes
Fixed Size		    2213936 bytes
Variable Size		  339740624 bytes
Database Buffers	   67108864 bytes
Redo Buffers		    8482816 bytes
Database mounted.

SQL> ALTER DATABASE NOARCHIVELOG;
Database altered.

SQL> ALTER DATABASE OPEN;
Database altered.
\end{lstlisting}

\subsection{System entlasten}
Bei unserer Instanz liegen die Trace-Files im angegebenen Verzeichnis in einem Unterordner \emph{trace}. Wie die Ausgabe von \emph{find ... | wc -l} zeigt, ist eine grosse Zahl dieser Dateien vorhanden:
\begin{lstlisting}
oracle_stud@hades:/app/oracle_stud/diag/rdbms/dbarc02/dbarc02/trace> find -iname '*.trc' | wc -l
3787
oracle_stud@hades:/app/oracle_stud/diag/rdbms/dbarc02/dbarc02/trace> rm *.trc

\end{lstlisting}

Unser Alert-Log ist mit ca. 100k noch recht kompakt, wir leeren es trotzdem.
\begin{lstlisting}
oracle_stud@hades:...dbarc02/trace> ls -lh alert_dbarc02.log 
-rw-r----- 1 oracle_stud oinstall 103K 2013-06-21 07:08 alert_dbarc02.log
oracle_stud@hades:...dbarc02/dbarc02/trace> echo -n "" > alert_dbarc02.log 
oracle_stud@hades:...dbarc02/trace> ls -lh alert_dbarc02.log 
-rw-r----- 1 oracle_stud oinstall 0 2013-06-21 07:17 alert_dbarc02.log
\end{lstlisting}

Wie der folgende Befehl zeigt, existiert auf dem Server Hades nirgends ein Verzeichnis \emph{bdump}:
\begin{lstlisting}
oracle_stud@hades:/app/oracle_stud/admin/dbarc02> find / -type d -iname 'bdump' 2>/dev/null 
oracle_stud@hades:/app/oracle_stud/admin/dbarc02>
\end{lstlisting}


\section{Export / Import einer Tabelle}
\subsection{Export}
Nach der Anmeldung führen wir gemäss Aufgabenstellung den Export-Befehl im entsprechenden Verzeichnis aus und überprüfen mittels ls, dass die Files erzeugt wurden (da der Datenbank-Name mit unserer SID übereinstimmt, verwenden wir gleich diese exportierte Variable):
\begin{lstlisting}
oracle_stud@hades:/app/oracle_stud/oradata/dbarc02/export> exp scott/tiger FILE=${ORACLE_SID} LOG=${ORACLE_SID}

Export: Release 11.2.0.1.0 - Production on Fri Jun 21 07:23:10 2013

Copyright (c) 1982, 2009, Oracle and/or its affiliates.  All rights reserved.


Connected to: Oracle Database 11g Enterprise Edition Release 11.2.0.1.0 - 64bit Production
With the Partitioning, OLAP, Data Mining and Real Application Testing options
Export done in US7ASCII character set and AL16UTF16 NCHAR character set
server uses AL32UTF8 character set (possible charset conversion)
. exporting pre-schema procedural objects and actions
. exporting foreign function library names for user SCOTT 
. exporting PUBLIC type synonyms
. exporting private type synonyms
. exporting object type definitions for user SCOTT 
About to export SCOTT's objects ...
. exporting database links
. exporting sequence numbers
. exporting cluster definitions
. about to export SCOTT's tables via Conventional Path ...
. . exporting table                           DEPT          4 rows exported
EXP-00091: Exporting questionable statistics.
EXP-00091: Exporting questionable statistics.
. . exporting table                            EMP         14 rows exported
EXP-00091: Exporting questionable statistics.
EXP-00091: Exporting questionable statistics.
. . exporting table                       SALGRADE          5 rows exported
EXP-00091: Exporting questionable statistics.
. exporting synonyms
. exporting views
. exporting stored procedures
. exporting operators
. exporting referential integrity constraints
. exporting triggers
. exporting indextypes
. exporting bitmap, functional and extensible indexes
. exporting posttables actions
. exporting materialized views
. exporting snapshot logs
. exporting job queues
. exporting refresh groups and children
. exporting dimensions
. exporting post-schema procedural objects and actions
. exporting statistics
Export terminated successfully with warnings.


oracle_stud@hades:/app/oracle_stud/oradata/dbarc02/export> ls -lh
total 28K
-rw-r--r-- 1 oracle_stud oinstall  24K 2013-06-21 07:23 dbarc02.dmp
-rw-r--r-- 1 oracle_stud oinstall 1.7K 2013-06-21 07:23 dbarc02.log
\end{lstlisting}

\subsection{Tabelle löschen}
Das folgende Listing zeigt das Löschen der \emph{emp}-Tabelle.

\begin{lstlisting}
oracle_stud@hades:/app/oracle_stud/oradata/dbarc02/export> sqlplus scott/tiger

SQL*Plus: Release 11.2.0.1.0 Production on Fri Jun 21 07:30:04 2013
Copyright (c) 1982, 2009, Oracle.  All rights reserved.

Error accessing PRODUCT_USER_PROFILE
Warning:  Product user profile information not loaded!
You may need to run PUPBLD.SQL as SYSTEM

Connected to:
Oracle Database 11g Enterprise Edition Release 11.2.0.1.0 - 64bit Production
With the Partitioning, OLAP, Data Mining and Real Application Testing options

SQL> DESC emp;
 Name					   Null?    Type
 ----------------------------------------- -------- ----------------------------
 EMPNO					   NOT NULL NUMBER(4)
 ENAME						    VARCHAR2(10)
 JOB						    VARCHAR2(9)
 MGR						    NUMBER(4)
 HIREDATE					    DATE
 SAL						    NUMBER(7,2)
 COMM						    NUMBER(7,2)
 DEPTNO 					    NUMBER(2)

SQL> DROP TABLE emp;
Table dropped.

SQL> DESC emp;
ERROR:
ORA-04043: object emp does not exist
\end{lstlisting}

\subsection{Tabelle zurückladen}
Wir führen den Import-Befehl wiederum gemäss Aufgabenstellung aus. Bei der ersten Frage (\glqq{}Import Data Only?\grqq{}), deren Antwort nicht in der Aufgabe vorgegeben ist, haben wir mit \glqq{}Yes\grqq{} geantwortet. 

Die Beantwortung mit \glqq{}No\grqq{} führt dazu, dass der Import-Befehl die Tabelle nicht erstellt und legiglich versucht, Datensätze einzufügen. Da die Tabelle zuvor aber entfernt wurde, schlägt dies fehl.

\begin{lstlisting}
oracle_stud@hades:/app/oracle_stud/oradata/dbarc02/export> imp scott/tiger

Import: Release 11.2.0.1.0 - Production on Fri Jun 21 07:43:07 2013
Copyright (c) 1982, 2009, Oracle and/or its affiliates.  All rights reserved.
Connected to: Oracle Database 11g Enterprise Edition Release 11.2.0.1.0 - 64bit Production
With the Partitioning, OLAP, Data Mining and Real Application Testing options

Import data only (yes/no): no > 
Import file: expdat.dmp > dbarc02.dmp
Enter insert buffer size (minimum is 8192) 30720> 

Export file created by EXPORT:V11.02.00 via conventional path
import done in US7ASCII character set and AL16UTF16 NCHAR character set
import server uses AL32UTF8 character set (possible charset conversion)

List contents of import file only (yes/no): no > 
Ignore create error due to object existence (yes/no): no > 
Import grants (yes/no): yes > 
Import table data (yes/no): yes > 
Import entire export file (yes/no): no > 
Username: scott
Enter table(T) or partition(T:P) names. Null list means all tables for user
Enter table(T) or partition(T:P) name or . if done: emp
Enter table(T) or partition(T:P) name or . if done: .

. importing SCOTT's objects into SCOTT
. importing SCOTT's objects into SCOTT
. . importing table                          "EMP"         14 rows imported
About to enable constraints...
Import terminated successfully without warnings.
\end{lstlisting}

\subsection{Tabelle prüfen}
Die Tabelle konnte erfolgreich wiederhergestellt werden:

\begin{lstlisting}
oracle_stud@hades:/app/oracle_stud/oradata/dbarc02/export> sqlplus scott/tiger

SQL*Plus: Release 11.2.0.1.0 Production on Fri Jun 21 07:45:46 2013
Copyright (c) 1982, 2009, Oracle.  All rights reserved.
Error accessing PRODUCT_USER_PROFILE
Warning:  Product user profile information not loaded!
You may need to run PUPBLD.SQL as SYSTEM

Connected to:
Oracle Database 11g Enterprise Edition Release 11.2.0.1.0 - 64bit Production
With the Partitioning, OLAP, Data Mining and Real Application Testing options

SQL> DESC emp;
 Name					   Null?    Type
 ----------------------------------------- -------- ----------------------------
 EMPNO					   NOT NULL NUMBER(4)
 ENAME						    VARCHAR2(10)
 JOB						    VARCHAR2(9)
 MGR						    NUMBER(4)
 HIREDATE					    DATE
 SAL						    NUMBER(7,2)
 COMM						    NUMBER(7,2)
 DEPTNO 					    NUMBER(2)
\end{lstlisting}


\subsection{View erstellen}
\subsubsection{Vergleich Möglichkeiten MySql}
\subsubsection{Vergleich Möglichkeiten MS-SQL}

\subsection{Rollen definieren}
\subsubsection{Vergleich Möglichkeiten MySql}
\subsubsection{Vergleich Möglichkeiten MS-SQL}

\subsection{Den Rollen Rechte zuweisen}
\subsubsection{Vergleich Möglichkeiten MySql}
\subsubsection{Vergleich Möglichkeiten MS-SQL}

\subsection{Den User Rollen zuweisen}
\subsubsection{Vergleich Möglichkeiten MySql}
\subsubsection{Vergleich Möglichkeiten MS-SQL}

\subsection{Überprüfen Sie die beiden Rollen}
\subsubsection{Vergleich Möglichkeiten MySql}
\subsubsection{Vergleich Möglichkeiten MS-SQL}

\section{Zugriffsrechte: Objekt- und Systemrechte}

\subsection{Objektrechte}
\subsubsection{Vergleich Möglichkeiten MySql}
\subsubsection{Vergleich Möglichkeiten MS-SQL}

\subsection{Systemrechte}
\subsubsection{Vergleich Möglichkeiten MySql}
\subsubsection{Vergleich Möglichkeiten MS-SQL}

\section{Views}
\subsection{Rechte auf Views}
\subsubsection{Vergleich Möglichkeiten MySql}
\subsubsection{Vergleich Möglichkeiten MS-SQL}

\subsection{DDL-Änderungen an den Basistabellen}
\subsubsection{Vergleich Möglichkeiten MySql}
\subsubsection{Vergleich Möglichkeiten MS-SQL}

\subsection{Updatable Views}
\subsubsection{Vergleich Möglichkeiten MySql}
\subsubsection{Vergleich Möglichkeiten MS-SQL}

\subsection{WITH CHECK OPTION}
\subsubsection{Vergleich Möglichkeiten MySql}
\subsubsection{Vergleich Möglichkeiten MS-SQL}

\section{Reflexion}

\end{document}